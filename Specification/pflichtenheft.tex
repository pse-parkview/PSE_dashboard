
\documentclass[parskip=full,11pt]{scrartcl}
\usepackage[utf8]{inputenc}

\title{Performance Dashboard for Continuous Benchmarking of HPC Libraries}
\author{Chingun Ariunbat, Maximilian Schik, Walter Alexander B\"ottcher,\\ Darius Schefer, Jamil Bagga}

% section numbers in margins:
\renewcommand\sectionlinesformat[4]{\makebox[0pt][r]{#3}#4}

\newcommand{\scenario}[2]{\textbf{Scenario name:} #1 \\
\textbf{Participating actor:} #2}

% header & footer
\usepackage{scrlayer-scrpage}
\lofoot{\today}
\refoot{\today}
\pagestyle{scrheadings}

\usepackage[sfdefault,light]{roboto}
\usepackage[T1]{fontenc}
\usepackage[english]{babel}
\usepackage[yyyymmdd]{datetime} % must be after babel
\renewcommand{\dateseparator}{-} % ISO8601 date format
\usepackage[colorlinks=true, linkcolor=blue]{hyperref}
\usepackage{amsmath} % for $\text{}$
\usepackage[nameinlink]{cleveref}
\crefname{figure}{Abb}{Abb}
\usepackage[section]{placeins}
\usepackage{xcolor}
\usepackage[nonumberlist]{glossaries}     % provides glossary commands
\usepackage{graphicx}
\hypersetup{
	pdftitle={Pflichtenheft},
	bookmarks=true,
}
\usepackage{csquotes}

\makenoidxglossaries

\makenoidxglossaries

\newglossaryentry{developer}
{
	name=developer,
	plural=developers,
	description={Person working on the project that is to be benchmarked}
}

\newglossaryentry{configuration}
{
	name=configuration,
	plural=configurations,
	description={A complete description of a \gls{visualization}. It contains all the necessary information except the benchmark data.}
}

\newglossaryentry{template}
{
	name=template,
	plural=templates,
	description={A partial configuration of a \gls{visualization}. It contains preconfigured values, but leaves others blank for the user to costumize.}
}

\newglossaryentry{visualization}
{
	name=visualization,
	plural=visualizations,
	description={A graphical representation of benchmark data.}
}

\newacronym{ci}{CI}{Continuous Integration}

\newacronym{json}{JSON}{JavaScript Object Notation}


\newcommand\urlpart[2]{$\underbrace{\text{\texttt{#1}}}_{\text{#2}}$}

\usepackage{pflichtenheft}

\begin{document}
\maketitle

\section{Introduction}
Glossary acronym example: \\
\acrshort{ci} \\
\acrlong{ci} \\
\acrfull{ci}

\begin{center}
\urlpart{http}{protocol}%
\texttt{://}%
\urlpart{web.io}{host}%
\texttt{/}%
\urlpart{index}{path}%
\texttt{?}%
\urlpart{argument=somevalue}{parameter}%
\texttt{\#}%
\urlpart{theAnchor}{fragment}
\end{center}

more placeholder

\section{Goals}
% Diese Section sollte kurz und knapp "fuer Manager" sein
% und auf eine Seite passen.

\subsection{Required}

\criterium{heading}{crt:length}

yet more placeholder

\criterium{Schnelle Weiterleitung Kurz- zu Lang-URL}{crt:fast}

\criterium{Authentifizieren mit E-Mail oder Facebook}{crt:login}

\criterium{Rechtlichte Vorgaben werden eingehalten}{crt:tmg}

template

\subsection{Optional}

\criteriumOptional{Authentifizieren mit Github}{crt:github}

\criteriumOptional{Seite mit Betreiberinfo}{crt:about}

template

\subsection{Limitation}

\criteriumNot{Keine Wahl Kurz-URL}{crt:no-choice}

template

\section{Functional Requirements}

\functionality{Schnelle Weiterleitung}{fnc:o1}
\fulfills{crt:fast}

template

\functionality{template}{fnc:login}
\fulfills{crt:login}
\fulfills{crt:github}

template

\functionality{Auf jeder Seite ist ein Link \enquote{Impressum}}{fnc:impressum-link}
\fulfills{crt:tmg}

template

\functionality{Auf jeder Seite ist ein Link \enquote{Datenschutz}}{fnc:datenschutz-link}
\fulfills{crt:tmg}

template

\functionality{Daten werden persistent gespeichert}{fnc:persistence}

template

\section{Nonfunctional Requirements}

\nonFunctionality{Modernes Design}{nfc:design}

template

\nonFunctionality{Persistenz}{nfc:persistence}

template

\nonFunctionality{Erweiterbarkeit}{nfc:extensibility}

template

\section{Product Data}

\productData{Benchmark Results (Name in progress)}{pdt:benchmark_results}
Format: JSON/CSV \\
Description: 
\begin{itemize}
	\item saved on server
	\item algorithm result data (time, storage, accuracy, convergence(?))
\end{itemize}

\productData{Git Histories}{pdt:git_histories}
Format: ??? (WIP) \\

\productData{\Glspl{template}}{pdt:template}
Format: JSON (?)


\section{Scenarios}

\scenario{Inspecting Last Change}{scn:inspect_last_change}
{Ted: \Gls{user}, CI: \Gls{benchmarking system}}
{Ted works on a project that has \emph{PROJECT NAME} set up. He makes changes on a performance critical component. After that he pushes his changes to the repository. The CI sends its benchmark results to the system, which stores it in a persistent way. Ted opens the webapp and selects a benchmark. He sees a list of all recent changes. The changes without benchmark data are greyed out. Ted selects his newest change. He selects a device to take the benchmark data from. The change appears in a list of selected changes. Ted selects the \enquote{Create New Plot} option. A popup appears. Ted chooses the dimension he wants to inspect. After he configured his plot he decides to save the \gls{template} for later use. He selects the \enquote{Save Template} option. He enters a name for the \gls{template} and selects the \enquote{Save} option. After that he selects the \enquote{Generate Plot} option. Ted gets redirected to a new site where he can inspect his plot. He decides to send this plot to a coworker. He selects the \enquote{Share} option and a link gets displayed. He copies the link and sends it to his coworker.}

\scenario{Comparing Benchmarks}{scn:comp_benchmarks}
{Greta: \Gls{user}}
{Greta opens the webapp. She selects a benchmark. She selects two benchmarks by first picking a specific change and then a specific device. She opens the configuration popup by selecting the \enquote{Create New Plot} option. She wants to use a previously created \gls{template}. She selects the \enquote{Use Template} option and chooses her \gls{template} from a list of available ones. The settings specified in the \gls{template} get applied to the current \gls{configuration}. Greta makes final adjustments and then selects the \enquote{Generate Plot} option. After that she gets redirected to a new site where she can inspect her plot. Ted also wants to download the plot for use in his publication. He selects the \enquote{Export} option. He gets to choose between a seleciton of filetypes. He picks his preferred one. A link gets displayed leading to a download with the selected filetype.}

\scenario{Performance Tracking}{scn:perf_tracking}
{CI: \Gls{benchmarking system}}
{The CI runs a specific benchmark for a specific change on a specific device. It sends a POST request to the system containing the results and the benchmark type, change identification and device name. The system receives the results and stores them in a persistent database. It recognizes that the benchmark performance has dropped by a significant factor. The system triggers a hook that sends a message to the developer slack channel informing about the performance drop. It also publishes a comment under the change on GitHub.}



\appendix


\printnoidxglossaries

\end{document}

\section{Use Cases}

\case{Visualize}
{initiated by User}
{\gls{configuration} is available}
{\begin{enumerate}
    \item The web app sends a request to the backend containing the \gls{configuration}.
    \item The backend fetches the specified data from a databank.
    \item The backend does the calculations specified in the \gls{configuration} (mean, median, standard deviation).
    \item The backend sends the data back to the webapp.
    \item The webapp takes the data and generates the plot specified in the \gls{configuration}.
    \item The user gets redirected to a new site where he can inspect the generated plot.
\end{enumerate}}
{The plot specified by the \gls{configuration} gets shown to the User.}
{Shouldn't take more than 10 seconds}

\bigskip

\case{Configure Visualization}
{initated by User}
{User selected the \enquote{Create New Plot} option}
{\begin{enumerate}
    \item A popup appears.
    \item The user chooses a plot type.
    \item The user chooses between certain options that are specific to the plot type.
\end{enumerate}}
{Results are displayed in the web app}
{}

\bigskip

\case{Inspect Change}
{initated by User}
{benchmark data for commit is available}
{\begin{enumerate}
    \item The user selects a single commit.
    \item The user initiates the \texttt{Configure Visualization} use case by selecting the \enquote{Create New Plot} option.
    \item Once the user is satisfied with his \gls{configuration}, he initiates the \texttt{Visualize} use case by selecting the \enquote{Create New Plot} option in the popup.
\end{enumerate}}
{Results are displayed in the web app}
{If data is present: no more than 30 seconds?}

\bigskip

\case{Compare commits}
{intiated by User}
{benchmark results for all selected commits are available}
{\begin{enumerate}
    \item The user selects multiple commits.
    \item The user initiates the \texttt{Configure Visualization} use case by selecting the \enquote{Create New Plot} option.
    \item Once the user is satisfied with his \gls{configuration}, he initiates the \texttt{Visualize} use case by selecting the "Generate New Plot" option in the popup.
\end{enumerate}}
{The server caches the results for later use (how?)}
{wip}

\bigskip

\case{Share Visualization}
{initiated by User}
{A \gls{visualization} has been generated}
{\begin{enumerate}
    \item The user selects the \enquote{share visualization} option.
    \item A link gets displayed.
    \item The link redirects any visitors to the same visualization.
\end{enumerate}} 
{Link which redirects to the visualization}
{}

\bigskip

\case{Export Visualization}
{initiated by User}
{A \gls{visualization} has been generated}
{\begin{enumerate}
    \item The user selects the \enquote{export visualization} option.
    \item A popup appears.
    \item The user chooses a filetype for the export.
    \item The user confirmes and downloads the visualization in the choosen file format.
\end{enumerate}} 
{The User is offered a download of an export of the visualization}
{Support for the filetypes png, pdf and pgf (what is the preferred latex format?)}

\bigskip

\case{Save Template}
{initiated by User, (maybe web browser as well? the cookies get stored on the web browser)}
{The user is in the \texttt{Configure Visualization} use case}
{The \texttt{Save Template} use case extends the \texttt{Configure Visualization} use case.
\begin{enumerate}
    \item The User selects the \enquote{save template} option.
    \item The User enters a name for the \gls{template}.
    \item The webapp stores the template locally (cookies).
\end{enumerate}} 
{\Gls{template} is stored on the server for later use (can't be stored because we have no authentification, cookies are possible tho)}
{Requires less than 1kB of storage in the database}

\bigskip

\case{Use Template}
{initiated by User (maybe web browser as well? the cookies get stored on the web browser)}
{The user is in the \texttt{Configure Visualization} use case and a \gls{template} is available locally}
{The \texttt{Use Template} use case extends the \texttt{Configure Visualization} use case.
\begin{enumerate}
    \item The User selects the \enquote{use template} option.
    \item User is shown a list of all available \glspl{template}.
    \item User selects a \gls{template} from the list.
    \item The current \gls{configuration} options get set to the values specified in the template.
\end{enumerate}} 
{The User's dashboard is set to the selected template}
{???}

\bigskip

\case{Post Benchmark Results}
{initiated by External Benchmarking System}
{The application was benchmarked}
{\begin{enumerate}
    \item The benchmarking system makes a POST request to the backend containing the new benchmark data in \acrshort{json} format.
    \item The backend converts the received data into the correct format.
    \item The backend stores the received data in a database.
\end{enumerate}} 
{The received performance data is stored in a database}
{???}

\bigskip

\case{Performance Tracking}
{communicates with External Webservice}
{New benchmark data has been posted to the backend}
{\begin{enumerate}
    \item The backend evaluates the performance of the new benchmark data.
    \item The backend compares the performance of the new benchmark with the performance of the corresponding benchmark of the last commit.
    \item The backend relays the results to a configured number of hooks.
    \item The hooks contact their external webservices according to how they have been configured.
\end{enumerate}}
{The server fires a POST Request to all webhook subscribers}
{}

\bigskip

\case{UseCaseTemplate}
{Actors}
{Entry cond.}
{\begin{enumerate}
    \item Flow 1
    \item Flow 2
\end{enumerate}} 
{Exit cond.}
{Quality Requirements(?)}

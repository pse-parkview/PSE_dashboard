\section{Class descriptions}

Class descriptions here


\subsection{Frontend}

\subsubsection{ui}

\label{f:1}
\class{AppComponent}
{Main app component.}
{\\}
{\\}

\label{f:2}
\class{HeaderComponent}
{Component managing the topbar.}
{\\}
{\\}

\label{f:3}
\class{SideBarComponent}
{Component managing the sidebar.}
{\\}
{\\}

\label{f:4}
\class{SideCurrentChosenCommitComponent}
{Component managing the list containing the chosen commits.}
{\\}
{\\}

\label{f:5}
\class{SideBenchmarkSelectComponent}
{Component managing the selection of the benchmark type.}
{
  \begin{itemize}
      \item \identifier{public void spawnBenchmarkSelectDialog()}{Spawns the \texttt{BenchmarkSelectionDialogComponent}.}
  \end{itemize}
}
{\\}

\label{f:6}
\class{SideBenchmarkCompareComponent}
{Component managing the button for comparing commits.}
{
  \begin{itemize}
      \item \identifier{public void spawnBenchmarkCompareDialog()}{Spawns the \texttt{BenchmarkCompareDialogComponent}.}
  \end{itemize}
}
{\\}

\label{f:7}
\class{SideLoadFromTemplateComponent}
{Component managing the button for comparing commits.}
{
  \begin{itemize}
      \item \identifier{public void spawnLoadFromTemplateDialog()}{Spawns the \texttt{LoadFromTemplateDialogComponent}.}
  \end{itemize}
}
{\\}

\label{f:8}
\class{SidePreviousPlotsComponent}
{Component managing the list of previously generated plots.}
{\\}
{\\}

\label{f:9}
\class{PlotCardComponent}
{Small card containing a previous plot.}
{\\}
{\\}

\label{f:10}
\class{AveragePerformanceComponent}
{Component managing the overview of the average performance.}
{\\}
{\\}

\label{f:11}
\class{GitHistoryComponent}
{Component managing the display of git history.}
{\\}
{\\}

\label{f:12}
\class{SingleBenchmarkPlotComponent}
{Component managing the presentation of a single benchmark plot.}
{\\}
{\\}

\label{f:13}
\class{BenchmarkComparisonPlotComponent}
{Component managing the presentation of a comparison of benchmarks.}
{\\}
{\\}

\label{f:14}
\class{ScatterPlotComponent}
{Component managing the generation and presentation of a scatter plot.}
{\\}
{\\}

\label{f:15}
\class{LinePlotComponent}
{Component managing the generation and presentation of a line plot.}
{\\}
{\\}

\label{f:16}
\class{BarPlotComponent}
{Component managing the generation and presentation of a bar plot.}
{\\}
{\\}

\label{f:17}
\class{SummaryChartComponent}
{Component managing the generation and presentation of a summary chart.}
{\\}
{\\}

\label{f:18}
\class{PlotConfigurationDialogComponent}
{Component managing the configuration panel, where the user can configure his plot.}
{
  \begin{itemize}
      \item \identifier{public void routeToPlotView()}{Reroutes the user to the plot view.}
  \end{itemize}
}
{\\}

\label{f:19}
\class{BenchmarkSelectDialogComponent}
{Component managing the dialog for the selection of the benchmark type.}
{
  \begin{itemize}
      \item \identifier{public void spawnPlotConfigurationDialog()}{Spawns the \texttt{PlotConfigurationDialogComponent}.}
  \end{itemize}
}
{\\}

\label{f:20}
\class{LoadFromTemplateDialogComponent}
{Component managing the dialog for loading configurations from templates.}
{
  \begin{itemize}
      \item \identifier{public void spawnPlotConfigurationDialog()}{Spawns the \texttt{PlotConfigurationDialogComponent}.}
  \end{itemize}
}
{\\}

\label{f:21}
\class{ErrorDialogComponent}
{Component managing the error dialog in case of exceptional behavior.}
{\\}
{\\}

\label{f:22}
\class{CookieConsentDialogComponent}
{Component managing the dialog asking the user about his cookie preferences.}
{\\}
{\\}

\label{f:23}
\class{BenchmarkCompareDialogComponent}
{Component managing the dialog asking the user about his cookie preferences.}
{
  \begin{itemize}
      \item \identifier{public void spawnPlotConfigurationDialog()}{Spawns the \texttt{PlotConfigurationDialogComponent}.}
  \end{itemize}
  }
{\\}


\subsubsection{datahandler}

\label{f:24}
\class{DataService}
{A Service that gets injected into UI Classes that need to talk to the backend. Subject to change during implementation.}
{
  \begin{itemize}
    \item \identifier{private final HttpClient http}{Within @angular/common/http and part of the Angular Framework.}
  \end{itemize}
}
{
  \begin{itemize}
    \item \identifier{public String[] getBranchNames()}{Gets a list of all branch names.}
    \item \identifier{public Commit[] getCommitHistory(String branch)}{Retrieves the entire commit history of the given branch.}
    \item \identifier{public T getBenchmarks<T extends Benchmarks>(Commit[] commits)}{Retrieves all benchmarks that have a commit given to them out of the given ones.}
  \end{itemize}
}

\label{f:25}
\interface{Commit}
{Encapsulates information regarding a commit.}
{
  \begin{itemize}
    \item \identifier{Date date()}{The date the commit was made.}
    \item \identifier{String commitMessage()}{Self explanatory.}
    \item \identifier{String author()}{Self explanatory.}
    \item \identifier{boolean hasBenchmark()}{Whether this commit has a benchmark associated to it.}
    \item \identifier{String sha()}{The identifying hash of the commit.}
    \item \identifier{String branch()}{The branch name of this commit.}
  \end{itemize}
}

\label{f:26}
\interface{Summary}
{Encapsulates information regarding the summary of a benchmark.}
{
  \begin{itemize}
    \item \identifier{Map<K, V> summary()}{The different values associated to their name or key.}
    \item \identifier{int amountTested()}{Self explanatory.}
  \end{itemize}
}

\label{f:27}
\interface{Benchmark}
{Encapsulates information regarding one Benchmark Result, Keep in mind that this is the front end representation as it currently is. May be subject to change.}
{
    \begin{itemize}
      \item \identifier{Commit commit()}{Corresponding commit of a benchmark}
      \item \identifier{String device()}{Device this benchmark was run on}
      \item \identifier{Summary summary()}{Summary of this benchmark.}
      \item \identifier{BenchmarkType type()}{The type of this benchmark.}
    \end{itemize}
}

\label{f:28}
\interface{BenchmarkType}
{Encapsulates information about a benchmark type, namely what its name is, and what is plottable. May be subject to change.}
{
    \begin{itemize}
      \item \identifier{String name()}{What the type is called.}
      \item \identifier{String[] validXKeys()}{Permissible keys of the benchmark result that can be used for the x axis.}
      \item \identifier{String[] validYKeys()}{Permissible keys of the benchmark result that can be used for the y axis.}
    \end{itemize}
}

\label{f:29}
\interface{BenchmarkComparison}
{Encapsulates information about the comparison between two Benchmark results.}
{
  \begin{itemize}
    \item \identifier{Benchmark[] benchmarks()}{The benchmark results that are compared.}
    \item \identifier{BenchmarkType comparisonType()}{Contains information like what the permissible key to compare for are.}
  \end{itemize}
}

\label{f:30}
\class{ConversionBenchmark : BenchmarkType}
{Implements BenchmarkType and describes what it is and what is plottable.}
{\\}{\\}
\label{f:31}
\class{ConversionComparison : BenchmarkType}
{Implements BenchmarkType and describes what it is and what is plottable.}
{\\}{\\}
\label{f:32}
\class{SpmvBenchmark : BenchmarkType}
{Implements BenchmarkType and describes what it is and what is plottable.}
{\\}{\\}
\label{f:34}
\class{SpmvComparison : BenchmarkType}
{Implements BenchmarkType and describes what it is and what is plottable.}
{\\}{\\}
\label{f:35}
\class{SolverBenchmark : BenchmarkType}
{Implements BenchmarkType and describes what it is and what is plottable.}
{\\}{\\}
\label{f:36}
\class{SolverComparison : BenchmarkType}
{Implements BenchmarkType and describes what it is and what is plottable.}
{\\}{\\}
\label{f:37}
\class{PreconditionerBenchmark : BenchmarkType}
{Implements BenchmarkType and describes what it is and what is plottable.}
{\\}{\\}
\label{f:38}
\class{PreconditionerComparison : BenchmarkType}
{Implements BenchmarkType and describes what it is and what is plottable.}
{\\}{\\}
\label{f:39}
\class{BlasBenchmark : BenchmarkType}
{Implements BenchmarkType and describes what it is and what is plottable.}
{\\}{\\}
\label{f:40}
\class{BlasComparison : BenchmarkType}
{Implements BenchmarkType and describes what it is and what is plottable.}
{\\}{\\}


\subsubsection{datahandler}

\label{f:41}
\class{PlotService}
{A service that gets injected into UI classes that need them, like Components that are display plots.}
{
  \begin{itemize}
    \item \identifier{private final DataService}{A dependency since the PlotService has to talk with the backend somehow.}
  \end{itemize}
}
{
  \begin{itemize}
    \item \identifier{public BubbleChartMultiSeries generateScatterPlot(Benchmark benchmark, String xKey, String yKey)}{Requsts and transforms plotting data into an object that can be directly used by the UI.}
    \item \identifier{public BubbleChartMultiSeries generateScatterPlot(BenchmarkComparison comparison, String xKey, String yKey)}{Requsts and transforms plotting data into an object that can be directly used by the UI.}
    \item \identifier{public MultiSeries generateLinePlot(Benchmark benchmark, String xKey, String yKey)}{Requsts and transforms plotting data into an object that can be directly used by the UI.}
    \item \identifier{public MultiSeries generateLinePlot(BenchmarkComparison comparison, String xKey, String yKey)}{Requsts and transforms plotting data into an object that can be directly used by the UI.}
    \item \identifier{public MultiSeries generateBarPlot(Benchmark benchmark, String xKey, String yKey)}{Requsts and transforms plotting data into an object that can be directly used by the UI.}
  \end{itemize}
}

\label{f:42}
\interface{PlotConfiguration}
{An interface describing the configurations needed to immediately recreate a plot.}
{
  \begin{itemize}
    \item \identifier{String benchmarkCommitSha()}{The commit referenced}
    \item \identifier{String benchmarkComparedCommitSha()}{The commit that was compared to}
    \item \identifier{String benchmarkType()}{The benchmark type to look for.}
    \item \identifier{String xKey()}{The key value that was plotted on the x axis.}
    \item \identifier{String yKey()}{The key value that was plotted on the y axis.}
  \end{itemize}
}

\subsubsection{cookies}
\label{f:43}
\class{CookieService}
{Handles the cookies used by the front end to save things like recently viewed plots and similar.}
{
  \begin{itemize}
    \item \identifier{private final ngx-cookie-service.CookieService cookieService}{The utility that manages cookies for us.}
  \end{itemize}
}
{
  \begin{itemize}
    \item \identifier{public void saveConsent(given: boolean)}{Saved a decision made by the user, whether he allows cookies or not.}
    \item \identifier{public boolean hasDecidedConsent()}{Whether the user has decided yet.}
    \item \identifier{public boolean getConsent()}{The current status on whether the app is allowed to save cookies or not.}
    \item \identifier{public void savePlotConfiguration(p: PlotConfiguration)}{Saves a configuration on the users browser.}
    \item \identifier{public void appendRecentPlotConfiguration(p: PlotConfiguration)}{Saves and appends a recently used plot configuration on the users browser.}
    \item \identifier{public PlotConfiguration[] getSavedPlotConfigurations()}{Gets the configurations actively saved by the user.}
    \item \identifier{public PlotConfiguration[] getRecentPlotConfigurations()}{Gets the configurations that were recently used by the user.}
  \end{itemize}
}

\subsection{Backend}
\subsubsection{benchmark}
\label{b:1}
\interface{BenchmarkResultStorage}
{This interface provides methods for the storage of benchmark results.}
{
  \begin{itemize}
      \item \identifier{void storeBenchmarkResults(results: List<BenchmarkResult>)}{Stores the given benchmark results in the database. If the benchmark result already exists, it gets replaced.}
  \end{itemize}
}

\label{b:2}
\class{BlasBenchmarkResult : BenchmarkResult}
{This is a benchmark result for the benchmarks of the BLAS format and type.}
{
  \begin{itemize}
    \item \identifier{private List<BlasDatapoint> datapoints}{List of datapoints belonging to this benchmark run.}
  \end{itemize}
}
{
  \begin{itemize}
      \item \identifier{public List<BlasDatapoint> getDatapoints()}{Getter for \texttt{datapoints}.}
  \end{itemize}
}

\label{b:3}
\class{BlasDatapoint}
{A single datapoint for the \texttt{BlasBenchmarkResult}, contains the problem description (n, r, m, k) and a list of operations.}
{
  \begin{itemize}
      \item \identifier{private int n}{n-value for this benchmark run.}
      \item \identifier{private int r}{r-value for this benchmark run.}
      \item \identifier{private int m}{m-value for this benchmark run.}
      \item \identifier{private int k}{k-value for this benchmark run.}
      \item \identifier{private List<Operation> operations}{List of operations, part of the benchmark.}
  \end{itemize}
}
{
  \begin{itemize}
      \item \identifier{public int getN()}{Getter for \texttt{n}.}
      \item \identifier{public int getR()}{Getter for \texttt{r}.}
      \item \identifier{public int getM()}{Getter for \texttt{m}.}
      \item \identifier{public int getK()}{Getter for \texttt{k}.}
      \item \identifier{public List<Operation> getOperations()}{Getter for \texttt{operations}.}
  \end{itemize}
}

\label{b:4}
\class{Component}
{A single component, part of \texttt{PreconditionerBenchmarkResult} and \texttt{SolverBenchmarkResult}.}
{
  \begin{itemize}
      \item \identifier{private String name}{Name of component.}
      \item \identifier{private double runtime}{Runtime for component.}
  \end{itemize}
}
{
  \begin{itemize}
      \item \identifier{public String getName()}{Getter for \texttt{name}.}
      \item \identifier{public double getRuntime()}{Getter for \texttt{runtime}.}
  \end{itemize}
}

\label{b:5}
\class{Conversion}
{A single conversion, part of \texttt{ConversionBenchmarkResult}.}
{
  \begin{itemize}
      \item \identifier{private String name}{Name of component.}
      \item \identifier{private double time}{Time to conversion.}
      \item \identifier{private bool completed}{Whether or not the benchmark has completed.}
  \end{itemize}
}
{
  \begin{itemize}
      \item \identifier{public String getName()}{Getter for \texttt{name}.}
      \item \identifier{public double getTime()}{Getter for \texttt{time}.}
      \item \identifier{public bool isCompleted()}{Getter for \texttt{completed}.}
  \end{itemize}
}

\label{b:6}
\class{ConversionBenchmarkResult : MatrixBenchmarkResult}
{This is a benchmark result for the benchmarks of the Conversion format and type.}
{ \\ }
{ \\ }

\label{b:7}
\class{ConversionDatapoint : MatrixDatapoint}
{A single datapoint for \texttt{ConversionBenchmarkResult}, contains the problem description for the matrix and a list of conversions.}
{
  \begin{itemize}
      \item \identifier{private List<Conversion> conversions}{List of conversions belonging to this benchmark run.}
  \end{itemize}
}
{
  \begin{itemize}
      \item \identifier{public int getConversions()}{Getter for \texttt{conversions}.}
  \end{itemize}
}

\label{b:8}
\class{Format}
{A single format, part of \texttt{SpmvBenchmarkResult}.}
{
  \begin{itemize}
      \item \identifier{private String name}{Name of format.}
      \item \identifier{private int storage}{}
      \item \identifier{private double time}{Time to format.}
      \item \identifier{private double maxRelativeNorm2}{}
      \item \identifier{private bool completed}{Whether or not this format has completed.}
  \end{itemize}
}
{
  \begin{itemize}
      \item \identifier{public String getName()}{Getter for \texttt{name}.}
      \item \identifier{public int getStorage()}{Getter for \texttt{storage}.}
      \item \identifier{public double getTime()}{Getter for \texttt{time}.}
      \item \identifier{public double getMaxRelativeNorm2()}{Getter for \texttt{maxRelativeNorm2}.}
      \item \identifier{public bool isCompleted()}{Getter for \texttt{completed}.}
  \end{itemize}
}

\label{b:9}
\class{MatrixBenchmarkResult : BenchmarkResult}
{Benchmark type that uses matrices.}
{
  \begin{itemize}
      \item \identifier{private List<MatrixDatapoint> datapoints}{List of datapoints belonging to this benchmark run.}
  \end{itemize}
}
{
  \begin{itemize}
      \item \identifier{public List<MatrixDatapoint> getDatapoints()}{Getter for \texttt{datapoints}.}
  \end{itemize}
}

\label{b:10}
\class{MatrixDatapoint}
{Single datapoint for a \texttt{MatrixBenchmarkResult}. Contains problem description (rows, columns, nonzeros).}
{
  \begin{itemize}
      \item \identifier{private String filename}{Filename for this benchmark run.}
      \item \identifier{private int rows}{Number of rows.}
      \item \identifier{private int columns}{Number of columns.}
      \item \identifier{private int nonzeros}{Number of non-zeros.}
  \end{itemize}
}
{
  \begin{itemize}
      \item \identifier{public String getFilename()}{Getter for \texttt{filename}.}
      \item \identifier{public int getRows()}{Getter for \texttt{rows}.}
      \item \identifier{public int getColumns()}{Getter for \texttt{columns}.}
      \item \identifier{public int getNonzeros()}{Getter for \texttt{nonzeros}.}
  \end{itemize}
}

\label{b:11}
\class{Operation}
{A single operation, part of \texttt{BlasBenchmarkResult}.}
{
  \begin{itemize}
      \item \identifier{private String name}{Name of operation.}
      \item \identifier{private double time}{}
      \item \identifier{private double flops}{}
      \item \identifier{private double bandwidth}{}
      \item \identifier{private bool completed}{Whether or not this operation has completed.}
  \end{itemize}
}
{
  \begin{itemize}
      \item \identifier{public String getName()}{Getter of \texttt{name}.}
      \item \identifier{public double getTime()}{Getter of \texttt{time}.}
      \item \identifier{public double getFlops()}{Getter of \texttt{flops}.}
      \item \identifier{public double getBandwidth()}{Getter of \texttt{bandwidth}.}
      \item \identifier{public bool getCompleted()}{Getter of \texttt{completed}.}
  \end{itemize}
}

\label{b:12}
\class{Preconditioner}
{A single preconditioner, part of \texttt{PreconditionerBenchmarkResult}.}
{
  \begin{itemize}
      \item \identifier{private String name}{Name of preconditioner.}
      \item \identifier{private List<Component> generateComponents}{}
      \item \identifier{private double generateTime}{}
      \item \identifier{private List<Component> applyComponents}{}
      \item \identifier{private double applyTime}{}
      \item \identifier{private bool completed}{Whether or not this preconditioner has completed.}
  \end{itemize}
}
{
  \begin{itemize}
      \item \identifier{private String getName()}{Getter for \texttt{name}.}
      \item \identifier{private List<Component> getGenerateComponents()}{Getter for \texttt{generateComponents}.}
      \item \identifier{private double getGenerateTime()}{Getter for \texttt{generateTime}.}
      \item \identifier{private List<Component> getApplyComponents()}{Getter for \texttt{applyComponents}.}
      \item \identifier{private double getApplyTime()}{Getter for \texttt{applyTime}.}
      \item \identifier{private bool isCompleted()}{Getter for \texttt{completed}.}
  \end{itemize}
}

\label{b:13}
\class{PreconditionerBenchmarkResult : MatrixBenchmarkResult}
{This is a benchmark result for the benchmarks of the Preconditioner format and type.}
{ \\ }
{ \\ }

\label{b:14}
\class{PreconditionerDatapoint : MatrixDatapoint}
{A single datapoint for \texttt{PreconditionerBenchmarkResult}, contains the problem description for the matrix and a list of preconditioners.}
{
  \begin{itemize}
      \item \identifier{private List<Preconditioner> preconditioners}{List of preconditioners belonging to this benchmark run.}
  \end{itemize}
}
{
  \begin{itemize}
      \item \identifier{public List<Preconditioner> getPreconditioners()}{Getter for \texttt{preconditioners}.}
  \end{itemize}
}

\label{b:15}
\class{Solver}
{A single Solver, part of \texttt{SolverBenchmarkResult}.}
{
  \begin{itemize}
      \item \identifier{private String name}{Name of solver.}
      \item \identifier{private List<double> recurrentResiduals}{}
      \item \identifier{private List<double> trueResiduals}{}
      \item \identifier{private List<double> implicitResiduals}{}
      \item \identifier{private List<double> iterationTimestamps}{}
      \item \identifier{private List<double> rhsNorm}{}
      \item \identifier{private List<double> residualNorm}{}
      \item \identifier{private List<Component> generateComponents}{}
      \item \identifier{private double generateTime}{}
      \item \identifier{private List<Component> applyComponents}{}
      \item \identifier{private double applyTime}{}
      \item \identifier{private int applyIterations}{}
      \item \identifier{private bool completed}{Whether or not this solver has completed.}
  \end{itemize}
}
{
  \begin{itemize}
      \item \identifier{public String getName()}{Getter for \texttt{name}.}
      \item \identifier{public List<double> getRecurrentResiduals()}{Getter for \texttt{recurrentResiduals}.}
      \item \identifier{public List<double> getTrueResiduals()}{Getter for \texttt{trueResiduals}.}
      \item \identifier{public List<double> getImplicitResiduals()}{Getter for \texttt{implicitResiduals}.}
      \item \identifier{public List<double> getIterationTimestamps()}{Getter for \texttt{iterationTimestamps}.}
      \item \identifier{public List<double> getRhsNorm()}{Getter for \texttt{rhsNorm}.}
      \item \identifier{public List<double> getResidualNorm()}{Getter for \texttt{residualNorm}.}
      \item \identifier{public List<Component> getGenerateComponents()}{Getter for \texttt{generateComponents}.}
      \item \identifier{public double getGenerateTime()}{Getter for \texttt{generateTime}.}
      \item \identifier{public List<Component> getApplyComponents()}{Getter for \texttt{applyComponents}.}
      \item \identifier{public double getApplyTime()}{Getter for \texttt{applyTime}.}
      \item \identifier{public int getApplyIterations()}{Getter for \texttt{applyIterations}.}
      \item \identifier{public bool getCompleted()}{Getter for \texttt{completed}.}
  \end{itemize}
}

\label{b:16}
\class{SolverBenchmarkResult : MatrixBenchmarkResult}
{This is a benchmark result for the benchmarks of the Solver format and type.}
{ \\ }
{ \\ }

\label{b:17}
\class{SolverDatapoint : MatrixDatapoint}
{A single datapoint for \texttt{SolverBenchmarkResult}, contains the problem description for the matrix and a list of solvers.}
{
  \begin{itemize}
      \item \identifier{private List<Solver> solvers}{List of solvers belonging to this benchmark run.}
  \end{itemize}
}
{
  \begin{itemize}
      \item \identifier{public int getSolvers()}{Getter for \texttt{solvers}.}
  \end{itemize}
}

\label{b:18}
\class{SpmvBenchmarkResult : MatrixBenchmarkResult}
{This is a benchmark result for the benchmarks of the SPMV format and type.}
{ \\ }
{ \\ }

\label{b:19}
\class{SpmvDatapoint : MatrixDatapoint}
{A single datapoint, contains the problem description for the matrix and a list of formats.}
{
  \begin{itemize}
      \item \identifier{private List<Format> formats}{List of formats belonging to this benchmark run.}
      \item \identifier{private Format optimal}{Optimal format for this benchmark run.}
  \end{itemize}
}
{
  \begin{itemize}
      \item \identifier{public List<Format> getFormats()}{Getter for \texttt{formats}.}
      \item \identifier{public Format getOptimal()}{Getter for \texttt{optimal}.}
  \end{itemize}
}

\subsubsection{database}

\label{b:20}
\class{BenchmarkResultDatabase : BenchmarkResultStorage}
{This class offers storage of benchmark results by using a database. Access to the database is provided by a \texttt{DatabaseHandler} object.}
{
  \begin{itemize}
      \item \identifier{private DatabaseHandler databaseHandler}{\texttt{DatabaseHandler} that is used to access a database.}
  \end{itemize}
}
{ \\ }

\label{b:21}
\interface{DatabaseHandler}
{Interface for accessing a database. It offers methods for storing, updating and retrieving commits and benchmark results}
{
  \begin{itemize}
      \item \identifier{public void updateCommits(List<Commit> commits)}{Updates existing commits in the database with the ones given as a parameter or adds them to the database if they dont exist yet. If a commit with the same sha as a given Commit already exists, the commit in the database gets replaced by the given commit.}
      \item \identifier{public void updateBenchmarkResults(List<BenchmarkResult> results)}{Updates existing benchmark results in the database with the ones given as a parameter or adds them to the database if they dont exist yet. If a benchmark result for the same commit, device, benchmark, problem setup and time already exists, it gets replaced.}
      \item \identifier{public BranchForBenchmark fetchBranch(String branch, Benchmark benchmark)}{Fetches all commits for a given branch and benchmark type and packs them into a branch object. The commits contain their corresponding benchmark results for every device available.}
      \item \identifier{public BenchmarkResult fetchBenchmarkResult(Commit commit, Device device, Benchmark benchmark)}{Fetches a single benchmark result for the given commit, device and benchmark type.}
  \end{itemize}
}

\label{b:22}
\class{HistoryDatabase}
{This class offers access to a git history stored in a database. Access to the database is provided by a \texttt{DatabaseHandler} object.}
{
  \begin{itemize}
      \item \identifier{private DatabaseHandler databaseHandler}{\texttt{DatabaseHandler} that is used to access a database.}
  \end{itemize}
}
{ \\ }

\label{b:23}
\class{LazyBenchmarkResult : BenchmarkResult}
{This class offers lazy loading of benchmark result. It loads the benchmark result from the database only once it is actually needed.}
{
  \begin{itemize}
      \item \identifier{private DatabaseHandler databaseHandler}{\texttt{DatabaseHandler} that is used to access a database.}
  \end{itemize}
}
{ \\ }

\label{b:24}
\class{PostgreSQLHandler : DatabaseHandler}
{\texttt{DatabaseHandler} for accessing a PostgreSQL database.}
{ \\ }
{ \\ }

\label{b:25}
\class{MissingBenchmarkResultException : Exception}
{Exception for handling missing benchmark results.}
{ \\ }
{ \\ }

\label{b:60}
\class{MissingCommitException : Exception}
{Exception for handling missing branches.}
{ \\ }
{ \\ }

\subsubsection{git}

\label{b:26}
\class{Benchmark}
{Contains information about a benchmark type.}
{
  \begin{itemize}
      \item \identifier{private String name}{Name of benchmark.}
  \end{itemize}
}
{
  \begin{itemize}
      \item \identifier{private String getName()}{Getter for \texttt{name}.}
  \end{itemize}
}

\label{b:27}
\interface{BenchmarkResult}
{Interface for representing a single benchmark result for a given benchmark type.}
{
  \begin{itemize}
      \item \identifier{public Commit getCommit()}{Returns the commit used for this benchmark.}
      \item \identifier{public Device getDevice()}{Returns the device used for this benchmark.}
      \item \identifier{public Benchmark getBenchmark()}{Returns the benchmark type.}
      \item \identifier{public Commit getSummaryValue()}{Returns the summary value for this benchmark.}
  \end{itemize}
}

\label{b:28}
\class{BranchForBenchmark}
{Class that represent a branch for a benchmark type. Benchmark is fixed since results can't be compared across benchmarks.}
{
  \begin{itemize}
      \item \identifier{private String name}{Name of branch.}
      \item \identifier{private Benchmark benchmark}{Benchmark type for this branch.}
      \item \identifier{private List<Commit> commits}{list of commits contained in this branch.}
  \end{itemize}
}
{
  \begin{itemize}
      \item \identifier{public String getName}{Getter for \texttt{name}.}
      \item \identifier{public Benchmark getBenchmark()}{Getter for \texttt{benchmark}.}
      \item \identifier{public Commit getCommit(String sha)}{Returns the commit for the given sha.}
      \item \identifier{public Commit toList()}{Returns this branch's commits.}
  \end{itemize}
}

\label{b:29}
\class{Commit}
{Class that represents a single commit for a given branch and benchmark type.}
{
  \begin{itemize}
      \item \identifier{private String sha}{Commit sha.}
      \item \identifier{private String message}{Commit message.}
      \item \identifier{private Date date}{Commit date.}
      \item \identifier{private List<Commit> parents}{Parent commits.}
      \item \identifier{private Map<Device, BenchmarkResult> benchmarkResultsByDevice}{A map from device types to benchmark results.}
  \end{itemize}
}
{
  \begin{itemize}
      \item \identifier{public String getSha()}{Getter for \texttt{sha}.}
      \item \identifier{public String getMessage()}{Getter for \texttt{message}.}
      \item \identifier{public Date getDate()}{Getter for \texttt{date}.}
      \item \identifier{public List<Commit> getParents()}{Getter for \texttt{parents}.}
      \item \identifier{public Map<Device, BenchmarkResult> getBenchmarkResultsByDevice()}{Getter for \texttt{benchmarkResultsByDevice}.}
  \end{itemize}
}

\label{b:30}
\class{Device}
{Class that contains information about a device.}
{
  \begin{itemize}
      \item \identifier{private String name}{Name of device.}
  \end{itemize}
}
{
  \begin{itemize}
      \item \identifier{public String getName()}{Getter for \texttt{name}.}
  \end{itemize}
}

\label{b:31}
\class{EmptyBenchmarkResult : BenchmarkResult}
{Null object for representing an empty or non existing benchmark result}
{ \\ }
{ \\ }

\label{b:32}
\interface{History}
{Interface that represents an entire git history. It allows for retrieving branches for a given benchmark type, therefore containing the results of the benchmarks run on this branch.}
{
  \begin{itemize}
      \item \identifier{public BranchForBenchmark getBranch(String name, Benchmark benchmark)}{Returns the branch with the given name for a given device and benchmark type.}
  \end{itemize}
}

\label{b:33}
\interface{RepositoryHandler}
{Interface that provides access to repository for fetching new commits. This allow for updating the history with new commits.}
{
  \begin{itemize}
      \item \identifier{public List<Commit> fetchGitHistory(String name)}{Returns the commits for a given branch as a List, since it doesn't contain any benchmark results.}
  \end{itemize}
}

\subsubsection{processing}

\label{b:34}
\interface{BlasPlotTransform}
{Interface for transforms using multiple \texttt{BlasBenchmarkResult}.}
{
  \begin{itemize}
      \item \identifier{public JSON transform(List<BlasBenchmarkResult> benchmarkResults)}{Transforms the benchmark data into a JSON containing the prepared values for plotting.}
  \end{itemize}
}

\label{b:35}
\class{BlasSinlgeScatterPlotTransform : BlasPlotTransform}
{Scatter plot using a single \texttt{BlasBenchmarkResult}.}
{ \\ }
{ \\ }

\label{b:36}
\class{BlasTwoScatterPlotTransform : BlasPlotTransform}
{Scatter plot using two \texttt{BlasBenchmarkResult}.}
{ \\ }
{ \\ }

\label{b:37}
\interface{ConversionPlotTransform}
{Interface for transforms using multiple \texttt{ConversionBenchmarkResult}.}
{
  \begin{itemize}
      \item \identifier{public JSON transform(List<ConversionBenchmarkResult> benchmarkResults)}{Transforms the benchmark data into a JSON containing the prepared values for plotting.}
  \end{itemize}
}

\label{b:38}
\class{ConversionSingleScatterPlotTransform : ConversionPlotTransform}
{Scatter plot using a single \texttt{ConversionBenchmarkResult}.}
{ \\ }
{ \\ }

\label{b:39}
\class{ConversionTwoScatterPlotTransform : ConversionPlotTransform}
{Scatter plot using two \texttt{ConversionBenchmarkResult}.}
{ \\ }
{ \\ }

\label{b:40}
\class{DataProcessor}
{This class takes care of processing and storing raw benchmark results.}
{ \\ }
{
  \begin{itemize}
      \item \identifier{public void storeBenchmarkResults(JSON results)}{Processes and stores benchmark results.}
      \item \identifier{public JSON transformBenchmarkResults(List<BenchmarkResult> results, PlotType plotType)}{Transforms the given benchmark results to a json containing the values for plotting the given plot type.}
  \end{itemize}
}

\label{b:41}
\enum{PlotType}
{Type of Plot}
{
  \begin{itemize}
      \item Blas
      \item Solver
      \item Spmv
      \item Conversion
      \item Preconditioner
  \end{itemize}
}

\label{b:42}
\interface{PreconditionerPlotTransform}
{Interface for transforms using multiple \texttt{PreconditionerBenchmarkResult}.}
{
  \begin{itemize}
      \item \identifier{public JSON transform(List<PreconditionerBenchmarkResult> benchmarkResults)}{Transforms the benchmark data into a JSON containing the prepared values for plotting.}
  \end{itemize}
}

\label{b:43}
\class{PreconditionerSingleScatterPlotTransform : PreconditionerPlotTransform}
{Scatter plot using a single \texttt{PreconditionerBenchmarkResult}.}
{ \\ }
{ \\ }

\label{b:44}
\class{PreconditionerTwoScatterPlotTransform : PreconditionerPlotTransform}
{Scatter plot using two \texttt{PreconditionerBenchmarkResult}.}
{ \\ }
{ \\ }

\label{b:45}
\class{PreconditionerSingleComponentBreakdownPlotTransform : PreconditionerPlotTransform}
{Breakdown of each component for a single \texttt{PreconditionerBenchmarkResult}.}
{ \\ }
{ \\ }

\label{b:46}
\interface{SolverPlotTransform}
{Interface for transforms using multiple \texttt{SolverBenchmarkResult}.}
{
  \begin{itemize}
      \item \identifier{public JSON transform(List<SolverBenchmarkResult> benchmarkResults)}{Transforms the benchmark data into a JSON containing the prepared values for plotting.}
  \end{itemize}
}

\label{b:47}
\class{SolverMultiRuntimePlotTransform : SolverPlotTransform}
{Plot showing the runtime for multiple \texttt{SolverBenchmarkResult}.}
{ \\ }
{ \\ }

\label{b:48}
\class{SolverMultiIterationCountsPlotTransform : SolverPlotTransform}
{Plot showing the interation counts for multiple \texttt{SolverBenchmarkResult}.}
{ \\ }
{ \\ }

\label{b:49}
\class{SolverMultiTimeToSolutionPlotTransform : SolverPlotTransform}
{Plot showing the time to solution for multiple \texttt{SolverBenchmarkResult}.}
{ \\ }
{ \\ }

\label{b:50}
\class{SolverMultiAchievedBandwidthPlotTransform : SolverPlotTransform}
{Plot showing the achieved bandwidth for multiple \texttt{SolverBenchmarkResult}.}
{ \\ }
{ \\ }

\label{b:51}
\class{SolverSingleConvergencePlotTransform : SolverPlotTransform : SolverPlotTransform}
{Plot showing the convergence for a single \texttt{SolverBenchmarkResult}.}
{ \\ }
{ \\ }

\label{b:52}
\class{SolverTwoSpeedupPlotTransform : SolverPlotTransform}
{Plot showing the speedup for two \texttt{SolverBenchmarkResult}.}
{ \\ }
{ \\ }

\label{b:53}
\interface{SpmvPlotTransform}
{Interface for transforms using multiple \texttt{SpmvBenchmarkResult}.}
{
  \begin{itemize}
      \item \identifier{public JSON transform(List<SpmvBenchmarkResult> benchmarkResults)}{Transforms the benchmark data into a JSON containing the prepared values for plotting.}
  \end{itemize}
}

\label{b:54}
\class{SpmvSinglePerformanceProfilePlotTransform : SpmvPlotTransform}
{Plot showing the performance for a single \texttt{SpmvBenchmarkResult}.}
{ \\ }
{ \\ }

\label{b:55}
\class{SpmvSingleScatterPlotTransform : SpmvPlotTransform}
{ScatterPlot for a single \texttt{SpmvBenchmarkResult}.}
{ \\ }
{ \\ }

\label{b:56}
\class{SpmvTwoSpeedupPlotTransform : SpmvPlotTransform}
{Plot showing the speedup for two \texttt{SpmvBenchmarkResult}.}
{ \\ }
{ \\ }

\subsubsection{rest}
\label{b:57}
\class{GitApiHandler : RepositoryHandler}
{Implements RepositoryHandler by using the GitHub Api.}
{ \\ }
{ \\ }

\label{b:58}
\interface{RestHandler}
{Interface that provides methods for handling POST and GET requests.}
{
  \begin{itemize}
      \item \identifier{public void handlePost(JSON json)}{Handles a POST request for uploading benchmark results.}
      \item \identifier{public void handleGetHistory(JSON json)}{Handles a GET request for retrieving commit history.}
      \item \identifier{public void handleGetBenchmarkResults(JSON json)}{Handles a GET request for retrieving benchmark results.}
  \end{itemize}
}

\label{b:59}
\class{SpringRestHandler}
{Class that implements a RestHandler using the Spring framework.}
{ \\ }
{ \\ }
